\section{Discussion}

\subsection{Method Performance Comparison}
Our results demonstrate that PLS-VIP consistently outperforms RWA across most experimental conditions. The superior performance of PLS-VIP (F = 633.63, p < 0.001, $\eta^2$ = 0.006) is particularly evident in more challenging scenarios, such as high-dimensional data and complex correlation structures. This advantage can be attributed to PLS-VIP's ability to handle multicollinearity and its robust variable importance calculation mechanism.

\subsection{Dimensionality Effects}
The strong impact of the number of predictors on performance ($\eta^2$ = 0.017) highlights a critical consideration in method selection. PLS-VIP's superior resilience to increased dimensionality suggests it may be particularly valuable for high-dimensional applications. The performance degradation observed in both methods with increasing predictors underscores the importance of careful variable selection in study design.

\subsection{Robustness to Data Characteristics}
Several key findings regarding method robustness emerged:
\begin{itemize}
    \item Both methods showed remarkable stability across data types (continuous vs. discrete)
    \item PLS-VIP demonstrated better resilience to correlation structure changes
    \item Sample size had surprisingly little impact on performance
\end{itemize}

These findings suggest that both methods can be reliably applied across various data conditions, though PLS-VIP offers greater stability in more challenging scenarios.

\subsection{Practical Implications}
Our findings have several important implications for practitioners:

\subsubsection{Method Selection Guidelines}
PLS-VIP is recommended for:
\begin{itemize}
    \item High-dimensional datasets ($J$ > 11)
    \item Scenarios with unknown or complex correlation structures
    \item Applications requiring robust performance across conditions
\end{itemize}

RWA might be preferred for:
\begin{itemize}
    \item Lower-dimensional problems ($J \leq 7$)
    \item Cases with well-understood correlation structures
    \item Situations where interpretability is paramount
\end{itemize}

\subsection{Limitations and Considerations}
Several limitations should be considered:
\begin{itemize}
    \item Performance variability increases with complexity for both methods
    \item The computational cost of PLS-VIP is higher than RWA
    \item Results may not generalize to all types of correlation structures
    \item The study assumes linear relationships between variables
\end{itemize}

\subsection{Future Research Directions}
This study suggests several promising avenues for future research:
\begin{itemize}
    \item Investigation of non-linear relationships
    \item Analysis of robustness to outliers and missing data
    \item Development of hybrid approaches combining strengths of both methods
    \item Exploration of computational efficiency improvements
    \item Extension to other types of correlation structures
\end{itemize} 