\section{Methods}

\subsection{Simulation Design}
We conducted a factorial simulation study to systematically evaluate the performance of PLS-VIP and RWA across diverse analytical conditions. The experimental design manipulated several key factors:

\begin{itemize}
\item \textbf{Number of Predictors ($J$)}: Three levels (2, 11, 20) representing low, medium, and high-dimensional scenarios.
\item \textbf{Correlation Structure ($\rho$)}: Three fixed values representing increasing levels of multicollinearity:
\begin{itemize}
\item Low correlation ($\rho = 0.0$)
\item Moderate correlation ($\rho = 0.5$)
\item High correlation ($\rho = 0.95$)
\end{itemize}
\item \textbf{Data Types}:
\begin{itemize}
\item Continuous variables (sampled from a multivariate normal distribution)
\item Ordinal variables (discretized into 5-point Likert scales)
\item Binary variables (dichotomized using median splits)
\end{itemize}
\item \textbf{Sample Sizes ($n$)}: Three levels (100, 200, 500).
\item \textbf{Effect Magnitudes}: Three levels representing weak ($\beta = 0.1$), moderate ($\beta = 0.3$), and strong ($\beta = 0.5$) predictor-outcome relationships.
\item \textbf{Noise Levels}: Three signal-to-noise ratios ($\text{SNR} = 0.5, 1.0, 2.0$).
\end{itemize}

\subsection{Data Generation Process}
For each combination of experimental factors, we generated synthetic datasets following these steps:
\begin{enumerate}
\item Generate a predictor correlation matrix with compound symmetry ($\rho$).
\item Sample predictor variables from a multivariate normal distribution.
\item For ordinal data, discretize predictors into 5-point Likert scales using quantile thresholds. For binary data, apply a median split.
\item Generate the outcome variable ($y$) as a weighted sum of relevant predictors plus Gaussian noise. The proportion of relevant predictors ($K$) was scaled as $K = \lceil 0.3 \cdot J \rceil$.
\item Replicate each condition 1000 times to ensure stable performance estimates.
\end{enumerate}

\subsection{Implementation}
We implemented the simulation study using the following tools:
\begin{itemize}
\item \textbf{PLS-VIP}: Custom implementation of VIP scores in \texttt{scikit-learn} (version 1.0.2).
\item \textbf{RWA}: Custom implementation in Python using principal components and variance decomposition.
\item \textbf{Data Generation}: Conducted using NumPy (version 1.21.0) for matrix operations.
\item \textbf{Statistical Analysis}: Performed using SciPy (version 1.7.0) for ANOVA and effect size calculations.
\end{itemize}

\subsection{Performance Metrics}
We evaluated method performance using the following metrics:
\begin{itemize}
\item \textbf{Top-$k$ Accuracy}: Proportion of correctly identified top-$k$ important variables.
\item \textbf{Mean Squared Error (MSE)}: Average squared difference between estimated and true importance scores.
\item \textbf{Rank Correlation}: Agreement between true and estimated rankings (Spearman and Kendall).
\end{itemize}

Statistical significance was assessed using factorial ANOVA with interaction effects, and partial eta-squared ($\eta^2$) was computed to quantify effect sizes. All analyses were conducted at a significance level of $\alpha = 0.05$.